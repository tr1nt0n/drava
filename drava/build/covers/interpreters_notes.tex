\documentclass[12pt]{article}
\usepackage{fontspec}
\usepackage[utf8]{inputenc}
\setmainfont{Bodoni 72 Book}
\usepackage[paperwidth=17in,paperheight=11in,margin=1in,headheight=0.0in,footskip=0.5in,includehead,includefoot,portrait]{geometry}
\usepackage[absolute]{textpos}
\TPGrid[0.5in, 0.25in]{23}{24}
\parindent=0pt
\parskip=12pt
\usepackage{nopageno}
\usepackage{graphicx}
\graphicspath{ {./images/} }
\usepackage{amsmath}
\usepackage{tikz}
\newcommand*\circled[1]{\tikz[baseline=(char.base)]{
            \node[shape=circle,draw,inner sep=1pt] (char) {#1};}}

\begin{document}

\vspace*{2\baselineskip}

\begingroup
\begin{center}
\huge FOREWORD
\end{center}
\endgroup

\begingroup
\begin{center}
\textbf{{\selectfont \setmainfont{Devanagari MT Bold}\selectfont द्रव} ( drava ):} flowing, fluid, dropping, dripping, trickling or overflowing. \\
\textbf{{\selectfont \setmainfont{Devanagari MT Bold}\selectfont ३} ( treeni ):} three. \\
\textbf{{\selectfont \setmainfont{Devanagari MT Bold}\selectfont गतय} ( gataya ):} motions. \\
\end{center}
\endgroup

\vspace*{4\baselineskip}

\begingroup
\begin{center}
\huge NOTES FOR THE INTERPRETER
\end{center}
\endgroup

\begingroup
\begin{center}
\textbf{\circled{1}} After temporary \textbf{accidentals}, cancellation marks are printed also in the following measure ( for notes in the same octave ) and, in the same measure, for notes in other octaves, but they are printed again if the same note appears later in the same measure, except if the note is immediately repeated. \textbf{\circled{2} Dynamics} apply only to the staff to which they are attached, to be achieved through manipulation of the register switches. However, if dynamics are accompanied by a \textbf{crescendo or decrescendo}, they apply to the entire organ, controlled by the instrument's \textbf{expression pedals}. \textbf{\circled{3} Grace notes} which \textbf{proceed} the note to which they are attached should be played at the \textbf{end} of the relevant beat. Grace notes \textbf{on} the note to which they are attached should be played at the \textbf{beginning} of the relevant beat, as fast as possible, afterwards returning to the main note. \textbf{\circled{4} A two line staff at the top of the system} is sometimes used to rhythmicise \textbf{tempo approximations}, wherein the top line indicates a common-practice \textbf{vivace} tempo, the bottom \textbf{grave}, and the intermediary space approximate positions between the two. In the absence of this staff, tempo indications are given in \textbf{beats per minute}. \textbf{\circled{5} Ornament articulations} should be interpreted according to the Baroque style, treating the note to which the articulation is attached as a tonic. \textbf{\circled{6} Jagged glissandi} indicate a \textbf{chromatic scale} from one point to another. \textbf{Straight glissandi} indicate a traditional white-key glissando.
\end{center}
\endgroup

\begingroup
\begin{center}
\textbf{\circled{7}} The interpreter reads \textbf{four staves}, wherein the \textbf{top three} represent \textbf{three keyboard manuals}, and the \textbf{bottom} represents the \textbf{foot pedals}. Each keyboard manual has a premeditated timbral profile, detailed below: \\
\circled{1} \textbf{Manual I} is bright, nasal, and present. It is recommended this timbre be achieved using any brass imitation register switches available to the individual organ. \\
\circled{2} \textbf{Manual II} is colorful to the point of fluorescence. It is recommended this timbre be achieved using secondary harmony register switches, especially of a perfect fifth or major third. \\
\circled{3} \textbf{Manual III} is hollow glass, akin to a crystallophone. It is recommended this timbre be achieved using any high violin, flute, and / or piccolo imitation register switches available to the individual organ.
\end{center}
\endgroup

\end{document}